\documentclass[11pt]{article}

\usepackage[latin1]{inputenc}
\usepackage{amsfonts}
\usepackage[paperwidth=17cm, paperheight=24cm, top=3cm, bottom=2.5cm, left=2.5cm, right=2.5cm]{geometry}

\begin{document}

\pagestyle{empty}
\begin{center}
{\Huge Propositions}\\[1cm]
{\Large accompanying the thesis}\\[0.5cm]
{\Large \textbf{Cellular Forces}} \\
{\Large \emph{Adhering, Shaping, Sensing and Dividing}}\\[1cm]
\begin{enumerate}
 \setlength{\itemsep}{10pt}
  \item The understanding of the stiffness variations in a (human) body is often reported as 
		having been resolved, but in reality is still in its infancy. \\
  \hfill \emph{Chapter 1 of this thesis}\\
  \item Cells on inverted micropillar arrays remain viable and enable the combination of traction force 
		measurements with super-resolution microscopy.\\
  \hfill  \emph{Chapter 2 of this thesis}\\
  \item Super resolution microscopy shows that the stress bearing state at a focal adhesion is an 
		order of magnitude larger than was previously measured. \\
  \hfill  \emph{Chapter 2 of this thesis}\\
  \item The extracellular shape of a cell depends on the orientation and contractility of the internal 
		actin cytoskeleton.\\
  \hfill  \emph{Chapter 3 of this thesis}\\
  \item The protein p130Cas is a mechanosensor that only localizes to focal adhesions in a 
		sufficiently stiff extracellular environment.\\
  \hfill  \emph{Chapter 4 of this thesis}\\
  \item During cell division, cellular pulling forces are released and cells push outwards while 
		progressing through mitosis.\\
  \hfill  \emph{Chapter 5 of this thesis}\\ 

\newpage
  \item Whether mechanosensing takes place on a cell-wide scale or at the length scale of an 
		adhesion can only be properly adressed when stiffness and molecular 
		stoichiometry are separately controlled. \\
  \hfill \emph{Trappmann et al.,} \textbf{Nat. Mater.} (2012)\\
  \item Durotaxis in fibroblasts is an example of global mechanosensing originating
		from contractile forces that increase with increasing extracellular stiffness. \\
  \hfill \emph{Trichet et al.,} \textbf{Proc. Natl. Acad. Sci. USA} (2012)\\
  \hfill \emph{Sochol et al.,} \textbf{Soft Matter} (2011)\\
  \item The development of organs-on-chips will not only revolutionize science and medical 
		treatments, but also cause a massive reorganization in the pharmaceutical industry. \\
  \hfill \emph{Van de Stolpe \& den Toonder,} \textbf{Lab Chip} (2013)\\
  \item Universities need to educate students and on the same premises researchers need to function 
		at the top of their fields: even though it has been tested multiple times, the Dutch government 
		cannot obtain these two goals for the price of one. \\
\end{enumerate}
\end{center}

\vfill
\flushright
  Hedde van Hoorn\\
  \today

\end{document}
